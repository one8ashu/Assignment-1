\documentclass[12pt]{article}
\usepackage{graphicx}
\graphicspath{{nitrr/}}
\title{\textbf{\underline{Assignment-1}}}
\date{26 January 2022\\Roll no.-21111058\\Semester-1\\\Large {Name-Shreyansh Srivastava}\\Subject-Biomedical Engineering\\Email-ID- 18shreyansh@gmail.com}
\begin{document}
   \maketitle
   	\begin{figure}[h]
    	\centering
   	    \includegraphics[scale=0.7]{nitrr.jpg}
   	\end{figure}
   \clearpage
      \begin{Acknowledgement}
         \\\Huge{\textbf{\underline{Acknowledgement}}}\\
         \\\large I would like to express my special thanks of grattitude to my teacher Dr. Saurabh Gupta who gave me the golden opportunity to do this wonderful assignment, which also helped in me doing a lot of research and i came to know about so many new things, I am really thankful to them.\\\\
         Secondly, I would also like to thank my parents and friends who helped me a lot in finalising the project within the limited time.\\\\
         At last, I would like to thanks all of them who helped me a lot in gathering information, collecting data and guiding me from time to time, despite of their hectic schedule.\\\\
         Thanking You\\
         Shreyansh Srivastava
    
   
      \end{Acknowledgement}
   \clearpage
   \tableofcontents
   \clearpage
   \section{{\textbf{\underline{{\large MEDICAL DEVICES} }}}}
    	
     A medical device is any device intended to be used for medical purposes, also a medical device is an instrument, apparatus, implant, machine, tool, in vitro reagent, or similar article that is to diagnose, prevent, mitigate, treat, or cure disease or other conditions, and, unlike a pharmaceutical or biological, achieves its purpose by physical, structural, or mechanical action but not through chemical or metabolic action within or on the body. \linebreak \linebreak
    \textbf{ FDA groups medical devices into three categories based on risk} \linebreak
     \textbf{Class I} products are low risk and do not require submission of data or information to FDA.\linebreak
     \textbf{Class II} products pose a moderate risk and are cleared via 510(k).\linebreak
     \textbf{Class III} products are high risk, innovative products requiring a Premarket Approval Application. 
     \paragraph{Here we giving some brief details about 5 medical devices:  }
      \clearpage
      \subsection{\underline{Alcohol Analyzer/ Breathealyzer}}
        \subsubsection{\underline{Introduction}}
        \paragraph{ A breathalyzer or breathalyser is a device for estimating Blood Alcohol Content(BAC). The name is a genericized trademark of the Breathalyzer brand name of instruments developed by inventor Robert Frank Borkenstein in the 1950s.\\There is a serious need to ensure that alcohol impaired drivers stay off the roads. It is estimated that one person is killed every 32 minutes and another person is injured every 26 seconds in alcohol related accidents. Highway deaths increased slightly from 41,717 deaths in 1999 to 41,812 in 2000. 40\% (16,725) involved alcohol, an increase from 38\% the previous year.\\Breath alcohol testers (BATs) depend on the blood to breath ratio. This ratio describes the relationship between the alcohol content of the breath and the alcohol content of blood at any given time. The accepted ratio of breath alcohol to blood alcohol is 2,100:1. This means 2,100 ml of a deep air lung sample contains the same amount of alcohol as 1 ml of blood.\\There are two main types of BATs used today:
      	\\\\\textbf{1.} The first type uses infrared light to detect alcohol content. This device passes a breath sample through a narrow band of infrared light set to a frequency absorbed by alcohol. The amount of infrared light not absorbed by the alcohol tells the concentration of alcohol in the breath.
        \\\\\textbf{2.} The  most commonly used BAT and second type of device uses fuel cells (which depends on chemical reactions). The alcohol in a person's breath is the energy for the fuel cell. The higher the concentration of breath alcohol, the more electricity that will be generated. The device measures the strength of the current to determine the breath/blood alcohol content (BAC).
         \\We mainly focus on fuel cell alcohol tester:}
           \begin{figure}[h]
          	\includegraphics[scale=0.6]{breath-alcohol-analyzer-500x500.jpg}
          	\caption{Breathe cell alcohol analyser}
          \end{figure}
        \\\subsubsection{\underline{Chemical reaction}}
        \paragraph{When the user exhales into a breath analyzer, any ethanol present in their breath is oxidized to acetic acid at the anode.\\
        \\$\[C_{2}H_{5}OH(g) + H_{2}O(l) = CH_{3}COOH(l) + 4H^{+}(aq) + 4e^{\--}\]
        \\$\[O_{2}(g) + 4H^{+}(aq) + 4e^{\--} = 2H_{2}O(l)\]
        \\The overall reaction is the oxidation of ethanol to acetic acid and water.
        \\$\[C_{2}H_{5}OH(l) + O_{2}(g) = CH_{3}COOH(aq) + H_{2}O(l)\] }
         \\\subsubsection{\underline{How it works?}}
        \paragraph{ 
        	\\\textbf{1.}Drivers are initially tested for alcohol impairment at the roadside with a screening device. 
        	\\\\\textbf{2.}The driver blows into a disposable mouthpiece for each test. The whole process takes about a minute for the device to record the result. Screening devices offer four result categories: “zero,” “pass,” “warn,” and “fail”. Anyone who fails the test is arrested and is required to perform an evidential breath test at a police station.
            \\\\\textbf{3.}A sample of the ambient air is tested as a blank check. This is followed by a check sample of an air/ethanol standard. This checks the calibration of the device. The concentration of alcohol in the standard sample is 35 μg/100 ml air. Two samples of breath are then taken from the motorist and tested, each separated by a sample of air. The test ends with a final air and standard check.
            \\\\\textbf{4.}If the results from the two actual samples differ by 15 \% or more of the lower reading or 5 μg, whichever is the greater, the device records an error message. 
        	\\\\\textbf{5.}The push is to manufacture portable devices that can collect evidence on the spot. To accomplish this, there is an increasing move towards manufacturing alcohol fuel cell devices.}
        \begin{figure}[h]
        	\centering
        	\includegraphics[scale=0.6]{fuelcell.jpg}
        	\caption{Fuel cell alcohol analyser}
        \end{figure}
        \clearpage    
        \subsection{\textbf{\underline{Camera pill}}}
         \begin{figure}[h]
        	\includegraphics[scale=0.4]{pill-cam.jpg}
        	\caption{Camera pil}
        \end{figure}
        \subsubsection{\underline{About}}
        \paragraph{Capsule endoscopy is a procedure that uses a tiny wireless camera to take pictures of your digestive tract. A capsule endoscopy camera sits inside a vitamin-size capsule you swallow. As the capsule travels through your digestive tract, the camera takes thousands of pictures that are transmitted to a recorder you wear on a belt around your waist.
        \\Capsule endoscopy helps doctors see inside your small intestine — an area that isn't easily reached with more-traditional endoscopy procedures. Traditional endoscopy involves passing a long, flexible tube equipped with a video camera down your throat or through your rectum.}
        \subsubsection{\underline{Pill components}}
        \begin{figure}[h]
        	\includegraphics[scale=0.4]{campil.jpg}
        	\caption{Camera pil components}
        \end{figure}
        \paragraph{The system consists of a sensor array, or electrodes, which are attached to the patient's abdomen, much like EKG leads or a Sensor Belt, worn around the abdominal area. These are connected to a data recorder which is worn by the patient during the study. The capsule, which is swallowed by the patient, is \textbf{26mm x 11mm} in size, and consists of an optical dome, a lens, several light emitting diodes, a semiconductor, transmitter, and an antenna. The disposable capsule is propelled physiologically through the entire GI tract, taking its most accurate images in the small bowel. Images recorded by the capsule camera are transmitted and stored on a data recorder worn by the patient. After the study, the images are downloaded onto a computer where the images are then viewed and interpreted by a specially trained gastroenterologist.
        }
        \subsubsection{\underline{Why it is used?}}
        \paragraph{\textbf{1.}  Find the cause of gastrointestinal bleeding
        	\\\textbf{2.}  Diagnose inflammatory bowel diseases, such as Crohn's disease
        	\\\textbf{3.}  Diagnose cancer
        	\\\textbf{4.}  Diagnose celiac disease
        	\\\textbf{5.}  Examine your esophagus
        	\\\textbf{6.}  Screen for polyps
        	\\\textbf{7.}  Do follow-up testing after X-rays or other imaging tests,if other tests are unclear.
        }
        \subsubsection{\underline{Risks related to Capsule endoscopy..}}
        \paragraph{The primary risk with capsule endoscopy is possible retention of the device in the small bowel. In patients who undergo the test to evaluate for bleeding, the risk is very low, approximately one to two percent. For patients with \textbf{Crohn's Diseas}e, the risk may increase to four to five percent.  Most cases of retention resolve spontaneously after a short delay in the passage of the capsule, and most patients have no symptoms whatsoever. Occasionally, medications are given to help facilitate passage. In rare instances, there is an abnormality in the small bowel which blocks the passage. In such a case, the capsule can be retrieved during an endoscopic procedure called a \textbf{double balloon enteroscopy}, or in unusual instances, by surgical resection.
        }
         
    \clearpage
    \subsection{\underline{Magnetic Resonance Imaging}}
        \begin{figure}[h]
       	\includegraphics[scale=0.4]{mri.jpg}
       	\caption{MRI Scanner}
       \end{figure}
     \subsubsection{\underline{About..}}
    \paragraph{Magnetic resonance imaging (MRI) is a type of scan that uses strong magnetic fields and radio waves to produce detailed images of the inside of the body.\\An MRI scanner is a large tube that contains powerful magnets. You lie inside the tube during the scan.\\An MRI scan can be used to examine almost any part of the body, including the:
    	\\\textbf{1.} brain and spinal cord
    	\\\textbf{2.} bones and joints
    	\\\textbf{3.} breasts
    	\\\textbf{4.} heart and blood vessels
    	\\\textbf{5.} internal organs, such as the liver, womb or prostate gland
        \\\textbf{6.} kidneys, spleen, pancreas, uterus, ovaries}
    \subsubsection{\underline{How does an MRI scan work?}}
    \paragraph{Most of the human body is made up of water molecules, which consist of hydrogen and oxygen atoms. At the centre of each hydrogen atom is an even smaller particle called a proton. \\Protons are like tiny magnets and are very sensitive to magnetic fields.	When you lie under the powerful scanner magnets, the protons in your body line up in the same direction, in the same way that a magnet can pull the needle of a compass.\\Short bursts of radio waves are then sent to certain areas of the body, knocking the protons out of alignment.When the radio waves are turned off, the protons realign. This sends out radio signals, which are picked up by receivers.\\These signals provide information about the exact location of the protons in the body.They also help to distinguish between the various types of tissue in the body, because the protons in different types of tissue realign at different speeds and produce distinct signals.\\In the same way that millions of pixels on a computer screen can create complex pictures, the signals from the millions of protons in the body are combined to create a detailed image of the inside of the body.
    }
    \subsubsection{\underline{Safety}}
    \paragraph{An MRI scan is a painless and safe procedure. You may find it uncomfortable if you have claustrophobia, but most people are able to manage it with support from the radiographer.\\Extensive research has been carried out into whether the magnetic fields and radio waves used during MRI scans could pose a risk to the human body.\\No evidence has been found to suggest there's a risk, which means MRI scans are one of the safest medical procedures available.But MRI scans may not be recommended in certain situations. \\For example, if you have a metal implant fitted, such as a pacemaker or artificial joint, you may not be able to have an MRI scan.They're also not usually recommended during pregnancy.
    }
    \subsubsection{\underline{New MRI just for Kids}}
     \begin{figure}[h]
    	\includegraphics[scale=0.4]{kids.png}
    	\caption{MRI Scanner designed for kids}
    \end{figure}
    \paragraph{ One of the most difficult challenges that MRI technicians face is obtaining a clear image, especially when the patient is a child or has some kind of ailment that prevents them from staying still for extended periods of time. As a result, many young children require anesthesia, which increases the health risk for the patient.\\By creating a pediatric coil made specifically for smaller bodies, the image can be rendered more clearly and quickly and will demand less MR operator skill. This will make MRIs cheaper, safer, and more available to children. \\Researchers are developing an optical tracking system that would be able to match and adapt the MRI pulses to changes in the patient’s pose in real time. \\Also making MRI scanner interactive for kids by relating it to some caartoon stories.
    }
    \clearpage
    \subsection{\underline{Stethoscope}}
    \begin{figure}[h]
    	\includegraphics[scale=0.4]{stetho.png}
    	\caption{Modern stethoscope}
    \end{figure}
    \subsubsection{\underline{About..}}
    \paragraph{The stethoscope was invented by Rene Laennec, a Frenchman, way back in 1816. However it is one of the oldest biomedical device. The stethoscope is an acoustic medical device for auscultation, or listening to internal sounds of an animal or human body. It typically has a small disc-shaped resonator that is placed against the skin, and one or two tubes connected to two earpieces.
    }
    \subsubsection{\underline{Types of stethoscope}}
    \paragraph{The various types of stethoscope are as follows:
   	    \\\textbf{1. Neonatal} The smallest for newborn patients, the smallest diameter of chest piece, some as small as as 2cm, allows for accurate auscultation without excess ambient noise.
    	\\\textbf{2. Infant}  it is a  like the pediatric and infant stethoscopes, but the diameter is in between the two. \\Around 2.7cm, the chest piece gives accurate auscultation for smaller patients.
    	\\\textbf{3. Pediatric} looks a lot like regular adult stethoscopes, but the main differences are in the size of chest piece and color. For a pediatric patient, it still might be difficult to position the chest piece since the chest size of children is much smaller than an adult.
    	\\\textbf{4. Fetal} The original fetal stethoscope, or fetoscope, was created in the 19th century and was basically an ear horn, called the Pinard Horn, it amplified the fetal heart beat into only one ear.
    	\\\textbf{5. Cardiology} A cardiology stethoscope is another that looks similar to a regular stethoscope, but the difference is that the cardiology version gives you far greater acoustic quality, and with it the ability to hear high and low frequency sounds more clearly.
    	\\\textbf{6. Veterinary} Many vets use exactly the same stethoscope as doctors with human patients, but this doesn’t have to be the case.Animals have particular needs and challenges that aren’t applicable to humans.For example, an animal’s fur makes a lot of noise that can interfere with auscultation.\\	The first design advantage is that most veterinary brands have long tubing, some up to 32 inches.
    	\\\textbf{7. Electronic} Acoustic stethoscopes have a sound problem, specifically volume. An electronic stethoscope solves this problem by electronically amplifying the sound obtained from the chest piece and converting it to electronic waves transmitted through specially designed circuitry.
    	\\\textbf{8. Amplifying} convert the acoustic sound of auscultation into electronic sound waves, which are then amplified and transmitted to the earpiece. Because the sound is electronic, you can do all sorts of beneficial things to it, such as reduce noise interference. \\It also allows you to wear headphones so the ear pieces don’t interfere with any hearing aids you might be wearing.
    	\\\textbf{9. Digitizing} Digital stethoscopes offer the chance to switch between analog and digital amplification. Switching modes allows you to pick the ideal setting for different types of diagnostic sounds.
    	\\\textbf{10. Adult stethoscope}}
    	\subsubsection{\underline{How a stethoscope works?}}
    	\paragraph{The disc and the tube of the stethoscope amplify small sounds such as the sound of a patient's lungs, heart and other sounds inside the body, making them sound louder. The amplified sounds travel up the stethoscope's tube to the earpieces that the doctor listens through. Heartbeats can easily be heard using a good stethoscope. Every time a person's heart beats it contracts and acts as a powerful pump, which circulates blood that carries oxygen and nutrients throughout the body.
    	}
    \clearpage
    \subsection{\underline{Glucose Monitoring Machine} }
     \begin{figure}[h]
    	\includegraphics[scale=0.6]{gluco.jpg}
    	\caption{Glucometer}
    \end{figure}
    \subsubsection{\underline{About..}}
    \paragraph{Glucose meters are a great tool, but sometimes you need to keep a closer eye on your blood sugar levels. That's where a device called a continuous glucose monitor (CGM) can help.
    }
    \subsubsection{\underline{What Does It Do?}}
    \paragraph{\textbf{1.} How much insulin you should take
    	\\\textbf{2.} An exercise plan that’s right for you
    	\\\textbf{3.} The number of meals and snacks you need each day
    	\\\textbf{4.} The correct types and doses of medications
    }
    \subsubsection{\underline{How does it works?}}
    \paragraph{Most glucometers today use an electrochemical method. Test strips contain a capillary that sucks up a reproducible amount of blood. The glucose in the blood reacts with an enzyme electrode containing glucose oxidase (or dehydrogenase). The enzyme is reoxidized with an excess of a mediator reagent, such as a ferricyanide ion, a ferrocene derivative or osmium bipyridyl complex. The mediator in turn is reoxidized by reaction at the electrode, which generates an electric current. The total charge passing through the electrode is proportional to the amount of glucose in the blood that has reacted with the enzyme. The coulometric method is a technique where the total amount of charge generated by the glucose oxidation reaction is measured over a period of time. The amperometric method is used by some meters and measures the electric current generated at a specific point in time by the glucose reaction. This is analogous to throwing a ball and using the speed at which it is travelling at a point in time to estimate how hard it was thrown. The coulometric method can allow for variable test times, whereas the test time on a meter using the amperometric method is always fixed. Both methods give an estimation of the concentration of glucose in the initial blood sample.
    }
    \subsubsection{\underline{Types of glucose meters?}}
    \paragraph{\textbf{1. Hospital glucose meters} are special glucose meters for multi-patient hospital use are now used. These provide more elaborate quality control records. Their data handling capabilities are designed to transfer glucose results into electronic medical records and the laboratory computer systems for billing purposes.
    \\\textbf{2. Blood testing with meters using test strips}
    \\\textbf{3. Noninvasive meters} are based on a technique for electrically pulling glucose through intact skin, and it was withdrawn after a short time owing to poor performance and occasional damage to the skin of users.
    \\\textbf{4. Continuous glucose monitors} this type of systems can consist of a disposable sensor placed under the skin, a transmitter connected to the sensor and a reader that receives and displays the measurements. The sensor can be used for several days before it needs to be replaced. The devices provide real-time measurements, and reduce the need for fingerprick testing of glucose levels. A drawback is that the meters are not as accurate because they read the glucose levels in the interstitial fluid which lags behind the levels in the blood. Continuous blood glucose monitoring systems are also relatively expensive.}
    \subsubsection{\underline{Future developments.}}
     \begin{figure}[h]
    	\includegraphics[scale=0.6]{G2.jpg}
    	\caption{ GlucoWatch G2 Biographer made by Cygnus Inc.}
    \end{figure}
    \paragraph{The GlucoWatch G2 Biographer made by Cygnus Inc. The device was designed to be worn on the wrist and used electric fields to draw out body fluid for testing. The device did not replace conventional blood glucose monitoring. One limitation was that the GlucoWatch was not able to cope with perspiration at the measurement site. Sweat must be allowed to dry before measurement can resume. Due to this limitation and others, the product is no longer on the market.\\
    The market introduction of noninvasive blood glucose measurement by spectroscopic measurement methods, in the field of near-infrared (NIR), by extracorporal measuring devices, has not been successful because the devices measure tissue sugar in body tissues and not the blood sugar in blood fluid. To determine blood glucose, the measuring beam of infrared light, for example, has to penetrate the tissue for measurement of blood glucose.\\
    Recent advances in cellular data communications technology have enabled the development of glucose meters that directly integrate cellular data transmission capability, enabling the user to both transmit glucose data to the medical caregiver and receive direct guidance from the caregiver on the screen of the glucose meter. The first such device, from Telcare, Inc., was exhibited at the 2010 CTIA International Wireless Expo, where it won an E-Tech award. This device then underwent clinical testing in the US and internationally.\\    
    In early 2014 Google reported testing prototypes of contact lenses that monitor glucose levels and alert users when glucose levels cross certain thresholds.
    }
    \bibliography{references}
    
      
\end{document}