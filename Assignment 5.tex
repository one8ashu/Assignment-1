\documentclass[12pt]{article}
\usepackage{graphicx}
\graphicspath{{nitrr/}}
\title{\textbf{\underline{Assignment-5}}}
\date{Roll no.-21111058\\Semester-1\\\Large {Name-Shreyansh Srivastava}\\Subject-Biomedical Engineering\\Email-ID- 18shreyansh@gmail.com}
\begin{document}
	\maketitle
	\begin{figure}[h]
		\centering
		\includegraphics[scale=0.7]{nitrr.jpg}
	\end{figure}
	\clearpage
	\begin{Acknowledgement}
		\\\Huge{\textbf{\underline{Acknowledgement}}}\\
		\\\large I would like to express my special thanks of grattitude to my teacher Dr. Saurabh Gupta who gave me the golden opportunity to do this wonderful assignment, which also helped in me doing a lot of research and i came to know about so many new things, I am really thankful to them.\\\\
		Secondly, I would also like to thank my parents and friends who helped me a lot in finalising the project within the limited time.\\\\
		At last, I would like to thanks all of them who helped me a lot in gathering information, collecting data and guiding me from time to time, despite of their hectic schedule.\\\\
		Thanking You\\
		Shreyansh Srivastava
	\end{Acknowledgement}
	\clearpage
	\paragraph{\textbf{\underline{\Large EMERGING TECHNOLOGIES IN HEALTHCARE}}
	\\\\With the change of time, new innovations came that will revolutionise the healthcare system of the world. 
    \\The development of health infrastructure also determines the overall development of the nation, and is one of the hot topics in the government's domain. The following are the domain that will revolutionise the health care system through the world.
    \\1. 3-D printing
    \\2. Virual reality (VR)
    \\3. Augmented reality (AR)
    \\4. Artificial intelligence (AI)
    \\5. Internet of things (IoT)
    \\6. Drones
    \\7. Robots
    \\8. Blockchain and many more.
    \\\\Following are the few technologies we are going to discuss in detail:}
    \section{\underline{Internet of Things (IoT)}}
    \paragraph{The internet of things, or IoT, is a system of interrelated computing devices, mechanical and digital machines, objects, animals or people that are provided with unique identifiers (UIDs) and the ability to transfer data over a network without requiring human-to-human or human-to-computer interaction.
    \\\\An IoT ecosystem consists of web-enabled smart devices that use embedded systems, such as processors, sensors and communication hardware, to collect, send and act on data they acquire from their environments. IoT devices share the sensor data they collect by connecting to an IoT gateway or other edge device where data is either sent to the cloud to be analyzed or analyzed locally.
    \\IoT can also make use of artificial intelligence (AI) and machine learning to aid in making data collecting processes easier and more dynamic. 
    \\(IoT) enabled devices have made remote monitoring in the healthcare sector possible, unleashing the potential to keep patients safe and healthy, and empowering physicians to deliver superlative care. It has also increased patient engagement and satisfaction as interactions with doctors have become easier and more efficient. }
    \section{\underline{Robots}}
    \paragraph{Robots in medicine help by relieving medical personnel from routine tasks, that take their time away from more pressing responsibilities, and by making medical procedures safer and less costly for patients. They can also perform accurate surgery in tiny places and transport dangerous substances.
    \\Robotic technologies appear in many areas that directly affect patient care. They can be used to disinfect patient rooms and operating suites, reducing risks for patients and medical personnel. They work in laboratories to take samples and the to transport, analyze, and store them. This is especially good news is you have ever had blood drawn by someone who had to try several times to find a "good vein." The robotic lab assistant can locate that vessel and draw the blood with less pain and anxiety for the patient. Robots also prepare and dispense medications in pharmacological labs. In larger facilities robotic carts carry bed linens and even meals from floor to floor, riding elevators and maneuvering through automatic doors. There are also "gears and wires" robotic assistants that help paraplegics move and can administer physical therapy.
    \\The ultimate question for robotics in healthcare is whether they will take jobs away from humans. There are several reasons why the machines will not replace their human counterparts. For one thing, most hospitals have less than 300 beds. They simply cannot afford the technology. The automated guided vehicles require a dedicated hall or floor tracks and the installation of navigation devices throughout the facilities. Other carts work with the help of a laser-drawn map of the hospital programmed into them that includes elevators, turns and automated doors. That process is also extremely expensive. But ultimately, robotic assistants cannot replace basic human contact. }
    \section{\underline{3-D printing}}
    \paragraph{Advances in 3D printing, also called additive manufacturing, are capturing attention in the health care field because of their potential to improve treatment for certain medical conditions. A radiologist, for instance, might create an exact replica of a patient’s spine to help plan a surgery; a dentist could scan a broken tooth to make a crown that fits precisely into the patient’s mouth. In both instances, the doctors can use 3D printing to make products that specifically match a patient’s anatomy.
    \\\\In healthcare, 3D bioprinting is used to create living human cells or tissue for use in regenerative medicine and tissue engineering.
    \\3D printing offers a completely new opportunity for the development and preparation of personalised medicines at both the pharmacy and industrial scale. Introducing 3D printers to pharmacies and hospitals would allow physicians, nurses, and pharmacists to form a dose and delivery system based on the patient’s body size, age, lifestyle, and sex. This would make medicine personal to the patient, and also save money and resources.
    \\\\3D bioprinting offers the potential to create functional, living, 3D human tissues of particular organs. These 3D tissues provide much more accurate mimicry to reality, resulting in much more predictive results for drug candidates, thereby reducing late-stage failures.}
	\section{\underline{Machine Learning \& Artificial Intelligence}}
	\paragraph{Artificial Intelligence and its subdivision Machine Learning is taking over the world right now, one of the best ML use for healthcare is a bot system that makes the treatment period much easier. Machine Learning for healthcare technologies provides algorithms with self-learning neural networks that are able to increase the quality of treatment by analyzing external data on a patient’s condition, their X-rays, CT scans, various tests, and screenings. Also, robots can be used in operations to much  more accuracy and can also increase the working time of hospitals, thus making health more accessible to public.
	\\However, currently deep learning is used for detecting cancer cells. The model is given tonnes of cancer cells pictures to memorize their looks. However, we are very far off from the total replacement of humans in medicine.
    \\\\The use of ML could boost the organizational side of the industry. Technology could easily take over these routine tasks such as claims processing, revenue cycle management, and clinical documentation and records management. Below are some of the uses of AI \& ML in healthcare:
    \\\textbf{1. } Diagnosis and disease identification
    \\\textbf{2. } Health records improvement
    \\\textbf{3. } The prediction of diabetes
    \\\textbf{4. } Predicting liver disease
    \\\textbf{5. } Finding the best cure
    \\\textbf{6. } Making diagnoses via image analysis
    \\\textbf{7. } Personalizing treatment
    \\\textbf{8. } Adjusting behavior
    \\\textbf{9. } Medical research and clinical trial improvement
    \\\textbf{10.} Leveraging crowdsourced medical data
    \\\textbf{11.} Epidemic control
    \\\textbf{12.} Artificial Intelligence Surgery
    \\\\Inspite of all these there are some drawbacks which need to be overcome by the data scientists:
    \\ 1.  Needs human surveillance
    \\ 2.  May overlook social variables
    \\ 3.  May leads to unemployment
    \\ 4.  Inaccuracies are still possible
    \\ 5.  Susceptible to security risks
    \\\\AI has doubtless potential to improve healthcare systems, automating tedious tasks can free up clinician schedules to allow for more patient interfacing.
    \\ Real time data can better and more rapidly inform diagnoses. Whether a patient or physician, lives everywhere are improving thanks to AI.}
	\section{\underline{Blockchain in healthcare}}
	\paragraph{Blockchain has a wide range of applications and uses in healthcare. The ledger technology facilitates the secure transfer of patient medical records, manages the medicine supply chain and helps healthcare researchers unlock genetic code.
    \\\\Benefits of blockchain is as follows:
    \\ 1.Security: data is highly secured in blockchain, because of its decentralised nature and the can be altered only through the consent of all nodes/ members in the system.
    \\ 3.Traceability: An immutable record of all transactions can reduce fraud and protect against liability.
    \\ 4.Business process speed: Automated smart contracts can reduce time of transactions because the process no longer requires manual oversight.
    \\ 5.Token value: Digital assets can hold virtual and real-world value, such as when a virtual token is used for a loyalty points programme.
    \\ 6.Confidentiality: Collaboration between organisations can occur without sharing sensitive information, e.g. individual medical records.
    \\ 7.Neutral and equal: No one company or individual owns the blockchain, encouraging trustworthiness and longevity of the system, also the system is decentralised e.g. if one of the founding parties leaves, the system will continue to work without them.
    \\\\However, with blockchain there are some demerits:
    \\ 1.Implementation and managinfg of blockchain requires deep knowledge, which needs experts to be hired.
    \\ 2.Blockchain can't go backwards, data is immutable, also data once inserted can't be deleted. Also using blockchain privacy is also at risk.
    \\ 3.Since blockchain is at its babystage thus cost of development isn't fixed.}
\end{document}