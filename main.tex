\documentclass{article}
\usepackage[utf8]{inputenc}
\usepackage{graphicx}
\graphicspath{{nitrr/}}
\title{Assignment 6\\
Semester-I\\
Biomedical Assignment}
\author{Submitted by: Shreyansh Srivastava}
\date{4th March 2022}

\begin{document}
	\maketitle
	\clearpage
	\begin{Acknowledgement}
		\\\Huge{\textbf{\underline{Acknowledgement}}}\\
		\\\large I would like to express my special thanks of gratitude to my teacher Dr. Saurabh Gupta who gave me the golden opportunity to do this wonderful assignment, which also helped in me doing a lot of research and i came to know about so many new things, I am really thankful to them.\\\\
		Secondly, I would also like to thank my parents and friends who helped me a lot in finalising the project within the limited time.\\\\
		At last, I would like to thanks all of them who helped me a lot in gathering information, collecting data and guiding me from time to time, despite of their hectic schedule.\\\\
		Thanking You\\
		Shreyansh Srivastava
	\end{Acknowledgement}
	\clearpage
	\paragraph{\Large \textbf{\centering \underline{Solutions to COVID-19 provided by biomedical engineers}}}
	\section{\Large \textbf{What do Biomedical Engineers do?}}
	\paragraph{The role of a Biomedical Engineer includes designing biomedical equipment and devices to aid the recovery or improve the health of individuals. It’s a role that requires excellent knowledge of computing, biology and engineering, an inventive nature, and good problem solving skills. It can also include creating and adapting medical equipment.\\
	With the hugely increased demand for ventilators and other equipment over the last few weeks, we’ve seen the important role that medical technology plays in patient care. Many people working in engineering have responded to the ongoing crisis by adapting their existing skills and equipment to help fight COVID-19.  But who are these Biomedical Engineers, and what do they do?\\
	This pandemic has helped to highlight some of the unseen profession that help to make our health services work, and show he positive impact that Biomedical Engineering in particular can have on people's lives. Here we highlight some of the contribution of biomedical engineers towards the healthcare sector.\\\\}
	\subsection{\textbf{Proper manufacturing of ventilators}}
	\paragraph{A ventilator is a device that supports or takes over the breathing process, pumping air into the lungs. People who stay in intensive care units (ICU) may need the support of a ventilator. This includes people with severe COVID-19 symptoms.\\
	Ventilators play an important role in saving lives, in both hospitals and ambulances. People who require long-term ventilation can also use them at home. The second wave of COVID-19 has a severe impact on the respiratory system, affecting the lungs severely thus the patient has to be put upon ventilator life support. There are different types of ventilators in the market:
	\\1. Face mask ventilator
	\\2. Mechanical ventilator
	\\3. Manual resuscitator bags
	\\4. Tracheostomy ventilators}
	\subsection{\textbf{PPE(Personal Protection Environment) kits}}
	\paragraph{The people most at risk of COVID-19 infection are those who are in close contact with a suspect/confirmed COVID-19 patient or who care for such patients.\\
	PPEs are protective gears designed to safeguard the health of workers by minimizing the exposure to a biological agent. Components of PPE are:
	\\1. Goggles 
	\\2. Face-shield 
	\\3. Masks
	\\4. Gloves 
	\\5. Coverall/Gowns (with or without aprons) 
	\\6. Head cover 
	\\7. Shoe cover\\\\
	According to experts “Compared with the general community, frontline health care workers had an 11.6-times higher risk of testing positive [for COVID-19] and those who reported that they had inadequate access to PPE had a 23\% higher risk.”}
	\subsection{\textbf{Antibodies test kits}}
	\paragraph{Antibody testing, also known as serology testing, is usually done after full recovery from COVID-19. Eligibility may vary, depending on the availability of tests. A health care professional takes a blood sample, usually by a finger prick or by drawing blood from a vein in the arm. Then the sample is tested to determine whether you've developed antibodies against the virus that causes COVID-19. The immune system produces these antibodies — proteins that are critical for fighting and clearing out the virus.\\
	If test results show that you have antibodies, it can mean that you have been infected with the COVID-19 virus in the past or you have antibodies after being vaccinated. Following tests are used for testing of COVID virus:
	\\\\ 1.\underline{RT-PCR test} also called a molecular test, this COVID-19 test detects genetic material of the virus using a lab technique called reverse transcription polymerase chain reaction (PCR). A fluid sample is collected with a nasal swab or a throat swab, or you may spit into a tube to produce a saliva sample.
	\\ 2.\underline{Antigen test} this COVID-19 test detects certain proteins in the virus. Using a nasal swab to get a fluid sample, antigen tests can produce results in minutes.}
	\subsection{\textbf{In-vitro diagnostics}}
	\paragraph{In vitro diagnostics are tests done on samples such as blood or tissue that have been taken from the human body. In vitro diagnostics can detect diseases or other conditions, and can be used to monitor a person’s overall health to help cure, treat, or prevent diseases.}
	\subsection{\textbf{Pulse oximeter}}
	\paragraph{Pulse oximetry is a noninvasive method for monitoring a person's oxygen saturation, it is a medical device that indirectly monitors the oxygen saturation of a patient's blood (as opposed to measuring oxygen saturation directly through a blood sample) and changes in blood volume in the skin, producing a photoplethysmogram that may be further processed into other measurements. It is used to measure person's peripheral oxygen saturation(SpO2).\\
	The new coronavirus that causes COVID-19 enters the body through the respiratory system, causing direct injury to a person's lungs via inflammation and pneumonia — both of which can negatively impact how well oxygen is transferred into the bloodstream. This oxygen impairment can occur at multiple stages of COVID-19, and not simply for critically ill patients placed on ventilators.\\
	It's why people may be wondering if an oximeter can help detect COVID-19 early. However, not everyone who tests positive for COVID-19 will develop low oxygen levels. There are people who may have a very uncomfortable bout with fever, muscle aches and GI upset at home, but never demonstrate low oxygen levels.}
	\subsection{\textbf{Oxygen concentrator}}
	\paragraph{An oxygen concentrator is an electricity-powered medical device that first takes in air, removes nitrogen, and then delivers a continuous source of concentrated oxygen to a patient requiring respiratory support. They are convenient for users and health-care workers because they are easy to move. One oxygen concentrator can simultaneously serve two adults and five children. During the second wave of coronavirus in India, there is a huge need of oxygen throughout the country, during these tough times these oxygen concentrators are a boon for the millions of families. but, unfortunately they are also present in the limited quantities which results in the end of many families.\\\\
	At last, I would like to say that the potential in humans is unlimited, thus we should come together and work for the betterment of the society.}
\end{document}
