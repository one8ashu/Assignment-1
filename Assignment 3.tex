\documentclass[12pt]{article}
\usepackage{graphicx}
\graphicspath{{nitrr/}}
\title{\textbf{\underline{Assignment-3}}}
\date{Roll no.-21111058\\Semester-1\\\Large {Name-Shreyansh Srivastava}\\Subject-Biomedical Engineering\\Email-ID- 18shreyansh@gmail.com}
\begin{document}
	\maketitle
	\begin{figure}[h]
		\centering
		\includegraphics[scale=0.7]{nitrr.jpg}
	\end{figure}
    \clearpage
    \begin{Acknowledgement}
    	\\\Huge{\textbf{\underline{Acknowledgement}}}\\
    	\\\large I would like to express my special thanks of grattitude to my teacher Dr. Saurabh Gupta who gave me the golden opportunity to do this wonderful assignment, which also helped in me doing a lot of research and i came to know about so many new things, I am really thankful to them.\\\\
    	Secondly, I would also like to thank my parents and friends who helped me a lot in finalising the project within the limited time.\\\\
    	At last, I would like to thanks all of them who helped me a lot in gathering information, collecting data and guiding me from time to time, despite of their hectic schedule.\\\\
    	Thanking You\\
    	Shreyansh Srivastava
    \end{Acknowledgement}
    \clearpage
    \paragraph{\textbf{\underline{\Large FUTURE OF HEALTHCARE}}}
    
    
    
    \section{\textbf{Why there is a need in evolution in healthcare?}}
    \paragraph{According to Darwin's theory of evolution if an organism needs to survive, then it must need to evolve with changing time and to survive it should be given necessary medical treatment, which can be only possible if the technologies in healthcare system is developing continuously because during a period of time new and more evolved viruses and bacteria keeps on coming. The history of healthcare dates back from centuries from the invention of ayurved by the \textbf{rishi munis}.
    \\\\In the current scenario of the world, the invention of various warfares like nuclear, biological, chemical there is a increase on healthcare system, also the birth of various virus, pathogens, etc pose a serious threat to mankind. Also the increase in population creaetes serious threat on current healthcare system. So to cope up with these problems the healthcare system should be evolved like use of machine learning in treatment of patients, adopting portable machines, performing more reseach in healthcare, including 3-D printers in colleges, storing data of patients in deceentralised server using blockchain, }


    
    \section{\textbf{Developments in healthcare}}
    \begin{figure}[h]
    	\includegraphics[scale=0.3]{trends.png}
    	\caption{Trends in healthcare untill next decade.}
    	\includegraphics[scale=0.2]{infographic.jpg}
    	\caption{Future developments.}
    \end{figure}
    \begin{figure}[h]
    
    	\includegraphics[scale=0.3]{Gartner20.png}
    	\caption{Hype cycle for AI.}
    \end{figure}
    \begin{figure}[h]
    	\includegraphics[scale=0.3]{Gartner21.png}
    	\caption{Hype cycle for emerging technologies.}
    \end{figure}
    \paragraph{The future of health will likely be driven by digital transformation enabled by radically interoperable data and open, secure platforms. Health is likely to revolve around sustaining well-being rather than responding to illness.\\\\
    Twenty years from now, cancer and diabetes could join polio as defeated diseases. We expect prevention and early diagnoses will be central to the future of health. The onset of disease, in some cases, could be delayed or eliminated altogether. Sophisticated tests and tools could mean most diagnoses (and care) take place at home.\\\\
    Based on emerging technology, we can be reasonably certain that digital transformation—enabled by radically interoperable data, artificial intelligence (AI), and open source, secure platforms—will drive much of this change.\\\\
    Today, wearable devices that track our steps, sleep patterns, and even heart rate have been integrated into our lives in ways we couldn’t have imagined just a few years ago. We expect this trend to accelerate. The next generation of sensors, for example, will move us from wearable devices to invisible, always-on sensors that are embedded in the devices that surround us.\\\\
    Many medtech companies are already beginning to incorporate always-on biosensors and software into devices that can generate, gather, and share data. Advanced cognitive technologies could be developed to analyze a significantly large set of parameters and create personalized insights into a consumer’s health. The availability of data and personalized AI can enable precision well-being and real-time microinterventions that allow us to get ahead of sickness and far ahead of catastrophic disease.\\\\
    Consumers—armed with this highly detailed personal information about their own health—will likely demand that their health information be portable. Consumers have grown accustomed to transformations that have occurred in other sectors, such as e-commerce and mobility. These consumers will demand that health follow the same path and become an integrated part of their lives—and they’ll vote with their feet and their wallets.\\\\
    Currently camera pills are also used for analysis, a pill that takes images of the inside of your digestive system – is being used to screen patients for bowel cancer from the comfort of their own homes. }




    
    \section{\textbf{How the healthcare is improving?}}
    \paragraph{ 
    	In the wake of the COVID-19 pandemic, we’re seeing the same shift in patients’ expectations for healthcare experiences. For example:
    	\\\textbf{1.} We’re seeing patients transition from in-person consultations to virtual visits. Just 8\% of patients had experienced a virtual doctor’s appointment pre-pandemic.
    	\\\textbf{2.} There’s been reduced reliance on “offline” patient-communications channels such as phone calls and letters, and increased demand for “always-on” services like patient apps and chatbots. For example, since the COVID-19 outbreak health app downloads have increased by 25\% , with more than 200 healthcare apps being added each day.
        \\\textbf{3.}	And amid working-from-home mandates, reduced childcare services and a highly volatile job market, data shows that convenience has become a top priority for today’s patients, with 80\% of patients now saying they select providers based on convenience alone.}
\end{document}