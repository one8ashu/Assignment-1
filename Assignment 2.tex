\documentclass[12pt]{article}
\usepackage{graphicx}
\graphicspath{{nitrr/}}
\title{\textbf{\underline{Assignment-1}}}
\date{Roll no.-21111058\\Semester-1\\\Large {Name-Shreyansh Srivastava}\\Subject-Biomedical Engineering\\Email-ID- 18shreyansh@gmail.com}
\begin{document}
	\maketitle
	\begin{figure}[h]
		\centering
		\includegraphics[scale=0.7]{nitrr.jpg}
	\end{figure}
	\clearpage
	\begin{Acknowledgement}
		\\\Huge{\textbf{\underline{Acknowledgement}}}\\
		\\\large I would like to express my special thanks of grattitude to my teacher Dr. Saurabh Gupta who gave me the golden opportunity to do this wonderful assignment, which also helped in me doing a lot of research and i came to know about so many new things, I am really thankful to them.\\\\
		Secondly, I would also like to thank my parents and friends who helped me a lot in finalising the project within the limited time.\\\\
		At last, I would like to thanks all of them who helped me a lot in gathering information, collecting data and guiding me from time to time, despite of their hectic schedule.\\\\
		Thanking You\\
		Shreyansh Srivastava
	\end{Acknowledgement}
    \clearpage
    \paragraph{\textbf{\underline{\Large EVOLUTION OF MODERN HEALTHCARE SYSTEM}}}
    
    \section{\textbf{Why there is a need in the evolution of health care system?}}
    \paragraph{In the currrent aspect of the world, there are various potential threats which can cause a major devastation to not even mankind but also to other species living on this earth such as:
    \\\textbf{1.}   Exponential growth in the population creates huge pressure on current health care system.
    \\\textbf{2.}   Widening of valley between rich and poor nations leads to the arbitrary behaviour of the former.
    \\\textbf{3.}   Increasaing risks of chemical warfare by the global superpowers.
    \\\textbf{4.}   Lack of investment by the governments in health industry.
    \\\textbf{5.}   During the current COVID-19 pandemic, even the even the biggest healthcare hubs like India, US, UK, France, etc becomes the prey.
    \\\textbf{6.}   The health facilities in some countries of Africa, Asia, etc is not better than worse. They still have to wait for decades to get supplies of vaccines just because of supremacy of powerful nations.
    \\\textbf{7.}   Lack of awareness among people, especially in rural areas about the importance of health.
    \\\textbf{8.}   Shortage of public hospitals with long waiting time often leading to untimely deaths, expensive treatment in private hospitals in addition to the growing lack of trust among patients.}
   
    \\\\\section{\textbf{How has the Indian healthcare system evolved in the last 75 years?}}
    \paragraph{Healthcare in India goes back to ancient times and has been documented in our scriptures like Vedas and Charak Samhita. Ayurveda or the “science of life” is believed to be imparted to the sages and rishis by the gods themselves. Well, this is mythology but the fact that Ayurveda has survived for ages might itself be considered as some sort of proof of the kind of medical knowledge our ancestors possessed.}\\
       \begin{figure}[h]
       	\includegraphics[scale=0.2]{ayurved.jpeg}
       	\caption{Ayurved, the traditional Indian practice of healing using herbs.}
       \end{figure}
     \paragraph{   Modern medicine was first brought in by the Portuguese but the first hospitals were built by the British and French. 19th century saw organised training in allopathy and other medical streams.\\
    Despite been ruled by foreigners, Indian healthcare has surely come a long way post-independence. Even though the healthcare infrastructure calls for a major overhaul, India has been able to put itself on the world map when it comes to medical tourism. Alternative treatments like Ayurveda, yoga and naturopathy have also seen a boom especially in the last decade or so.}
 
     \\\\\section{Effect of COVID-19 pandemic of India's healthcare.}
    \paragraph{With the COVID-19 pandemic putting even the world’s most advanced healthcare systems to the test, India’s healthcare system has also been disturbed. While usually sufficient, healthcare in India found itself on its knees by the ferocious second wave of COVID-19 in April 2021.\\
    A devastated Indian healthcare system infrastructure was brutally exposed by the lack of oxygen and drugs required for the treatment of COVID-19 in India. Additionally, the lack of awareness regarding healthcare insurance made it very difficult for the ordinary person to receive the full extent of in-patient care for COVID-19.\\
    However, not all news was bleak. The silver lining in the whole situation was that private Indian healthcare companies took the initiative and have been delivering the government with all of the resources it requires, including testing, isolation beds for treatment, medical personnel, and equipment at government COVID-19 hospitals, as well as home healthcare.\\
    The way the healthcare sector in India managed the second wave of COVID-19 is primarily thanks to the advanced healthcare education imparted at the college level to all students.}
    \\\\\section{Use of modern technologies in healthcare system...}
    \\\subsection{Cloud computing}
    \\\paragraph{Cloud computing provides a secure infrastructure to hospitals, medical practices, insurance companies, and research facilities. The main objective behind it is to improve computing resources at lower initial capital outlays. Also, cloud computing can reduce the barriers to innovation and modernization of healthcare systems and applications. It ultimately results in making the overall health data management system more flexible and scalable.\\
    Earlier, electronic health records, which enabled customization, required highly-skilled programmers and IT professionals to develop the needed customized solution. On the contrary, cloud-based solutions are fully customizable with built-in features and care plans. Cloud systems are easily accessible throughout the world, doctor can get experience from the operations throughout the globe. The following characteristics of the cloud address these challenges:
    \\\\\textbf{On-demand service:} The resources are provided immediately without any human intervention.
    \\\textbf{Resource Pooling:} Multiple users can have access to cloud services at the same time.
    \\\textbf{Elasticity:} It is possible to add, remove or upgrade as per the organizational requirements.
    \\\textbf{Broad Network Access:} A wide range of network accessibility is provided from any location at any time.
    \\\textbf{Measured Service:} Clients need to pay only for what they utilize.}
          \begin{figure}[h]
          	\includegraphics[scale=0.2]{cloud.jpg}
          	\caption{Cloud computing facilitates healthcare system.}
          \end{figure}
    \\\subsection{Machine Learning and Artificial Intelligence}
          \begin{figure}[h]
          	\includegraphics[scale=0.2]{ml.jpg}
          	\caption{}
          \end{figure}
    \\\paragraph{Machine Learning for healthcare technologies provides algorithms with self-learning neural networks that are able to increase the quality of treatment by analyzing external data on a patient’s condition, their X-rays, CT scans, various tests, and screenings.Following are the characterictics of ML and AI:
    \\\textbf{1.}Robot-assisted surgery
    \\\textbf{2.}Virtual nursing assistants
    \\\textbf{3.}Fraud detection
    \\\textbf{4.}Administrative workflow
    \\\textbf{5.}Dosage error reduction
    \\\textbf{6.}Connected machines
    \\\textbf{7.}Clinical trial participation
    \\\textbf{8.}Preliminary diagnosis
    \\\textbf{9.}Automated image diagnosis
    \\\textbf{10.}Cybersecurity
    \\\textbf{11.}Optimizing electronic records}
    \\\\\section{Future development in healthcare structure in the world...}
    \paragraph{Future planings were going on making the healthcare devices more portable and handy by connecting it to a centraliseed server and a probe used to take the readings. Thus healthcare workers can easily take it to remote areas for testing and upload it to centralised for the results.\\
    Also, through cloud computing doctors can treat the patients on sitting at distant places, students at medical colleges can also get vast knowledge on seeing live operations, 3-D models of various organs, virtual seminars, etc. Machine learning can also be used to treat patients without the involvement of doctors.\\
    The future of health that we envision is only about 20 years off, but health in 2040 will be a world apart from what we have now. Based on emerging technology, we can be reasonably certain that digital transformation—enabled by radically interoperable data, artificial intelligence (AI), and open, secure platforms—will drive much of this change. Unlike today, we believe care will be organized around the consumer, rather than around the institutions that drive our existing health care system.\\
    
    By 2040 (and perhaps beginning significantly before), streams of health data—together with data from a variety of other relevant sources—will merge to create a multifaceted and highly personalized picture of every consumer’s well-being. Today, wearable devices that track our steps, sleep patterns, and even heart rate have been integrated into our lives in ways we couldn’t have imagined just a few years ago. We expect this trend to accelerate. The next generation of sensors, for example, will move us from wearable devices to invisible, always-on sensors that are embedded in the devices that surround us.\\
    
    Many medtech companies are already beginning to incorporate always-on biosensors and software into devices that can generate, gather, and share data. Advanced cognitive technologies could be developed to analyze a significantly large set of parameters and create personalized insights into a consumer’s health. The availability of data and personalized AI can enable precision well-being and real-time microinterventions that allow us to get ahead of sickness and far ahead of catastrophic disease.\\
    
    Consumers—armed with this highly detailed personal information about their own health—will likely demand that their health information be portable. Consumers have grown accustomed to transformations that have occurred in other sectors, such as e-commerce and mobility. These consumers will demand that health follow the same path and become an integrated part of their lives—and they’ll vote with their feet and their wallets.  }
    \\\\\section{Why does the future of health matters?}
    \paragraph{Nothing is more important than our health. All of us interact with the health care system to varying degrees, and we will continue to interact with it throughout our lives. The cost of health care affects individuals, families, and employers as well as local, state, and central budgets.\\\\
   	Health care consumers typically interact with the health system only when they are sick or injured. But the future of health will be focused on well-being and prevention rather than treatment. Greater emphasis on well-being and identifying health risks earlier will result in fewer and less severe diseases, which will reduce health care spending, allowing the reinvestment of this well-being dividend to expand the benefits to the broad population. Along with helping to improve the well-being of individuals, health care stakeholders will also work to improve population health.\\\\
   	In response to this shifting health landscape, traditional jobs we know today will undergo change. Health will be monitored continuously so that risks can be identified early. Rather than assessing patients and treating them, the primary focus will be on sustaining well-being by providing consumers ongoing advice and support.\\\\
   	We don’t expect disease to have been eliminated entirely by 2040, but the use of actionable health insights—driven by interoperable data and smart AI—could help identify illness early, enable proactive intervention, and improve the understanding of disease progression. This can allow us to avoid many of the catastrophic expenses we have today. Technology might also help break down barriers such as cost and geography that can limit access to health care providers and specialists.\\\\
   	Health systems, health plans, and life sciences companies have begun to shift some of their focus to wellness, but the overall system remains focused on sick care.}
   \clearpage
       \paragraph{\textbf{\underline{\Huge Reference}}
   	    \centering
   	    \\\large \\1.https://www2.deloitte.com
   	    \\\\2.https://www.google.com
   	    \\\\3.https://thefutureishere.economist.com/healthcare
   	    \\\\4.https://www.leewayhertz.com/cloud-computing-in-healthcare/
   	    \\\\5.www.quora.com}
  
\end{document}