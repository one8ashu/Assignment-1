\documentclass[12pt]{article}
\usepackage{graphicx}
\graphicspath{{nitrr/}}
\title{\textbf{\underline{Assignment-4}}}
\date{Roll no.-21111058\\Semester-1\\\Large {Name-Shreyansh Srivastava}\\Subject-Biomedical Engineering\\Email-ID- 18shreyansh@gmail.com}
\begin{document}
	\maketitle
	\begin{figure}[h]
		\centering
		\includegraphics[scale=0.7]{nitrr.jpg}
	\end{figure}
	\clearpage
	\begin{Acknowledgement}
		\\\Huge{\textbf{\underline{Acknowledgement}}}\\
		\\\large I would like to express my special thanks of grattitude to my teacher Dr. Saurabh Gupta who gave me the golden opportunity to do this wonderful assignment, which also helped in me doing a lot of research and i came to know about so many new things, I am really thankful to them.\\\\
		Secondly, I would also like to thank my parents and friends who helped me a lot in finalising the project within the limited time.\\\\
		At last, I would like to thanks all of them who helped me a lot in gathering information, collecting data and guiding me from time to time, despite of their hectic schedule.\\\\
		Thanking You\\
		Shreyansh Srivastava
	\end{Acknowledgement}
	\clearpage
	\paragraph{\textbf{\underline{\Large DISRUPTIVE INNOVATIONS IN HEALTHCARE}}
	\\\\As time goes on there is a significance increase in the development of medical instruments so as to cope with various diseases, viruses, bacteria, etc, that are present or will come in the future. 
    \\The development of health infrastructure also determines the overall development of the nation, and is one of the hot topics in the government's domain.}
	\section{\underline{Machine Learning \& Artificial Intelligence}}
	\paragraph{Artificial Intelligence and its subdivision Machine Learning is taking over the world right now, one of the best ML use for healthcare is a bot system that makes the treatment period much easier. Machine Learning for healthcare technologies provides algorithms with self-learning neural networks that are able to increase the quality of treatment by analyzing external data on a patient’s condition, their X-rays, CT scans, various tests, and screenings. Also, robots can be used in operations to much  more accuracy and can also increase the working time of hospitals, thus making health more accessible to public.
	\\However, currently deep learning is used for detecting cancer cells. The model is given tonnes of cancer cells pictures to memorize their looks. However, we are very far off from the total replacement of humans in medicine.
    \\\\The use of ML could boost the organizational side of the industry. Technology could easily take over these routine tasks such as claims processing, revenue cycle management, and clinical documentation and records management. Below are some of the uses of AI \& ML in healthcare:
    \\\textbf{1. } Diagnosis and disease identification
    \\\textbf{2. } Health records improvement
    \\\textbf{3. } The prediction of diabetes
    \\\textbf{4. } Predicting liver disease
    \\\textbf{5. } Finding the best cure
    \\\textbf{6. } Making diagnoses via image analysis
    \\\textbf{7. } Personalizing treatment
    \\\textbf{8. } Adjusting behavior
    \\\textbf{9. } Medical research and clinical trial improvement
    \\\textbf{10.} Leveraging crowdsourced medical data
    \\\textbf{11.} Epidemic control
    \\\textbf{12.} Artificial Intelligence Surgery
    \\\\Inspite of all these there are some drawbacks which need to be overcome by the data scientists:
    \\ 1.  Needs human surveillance
    \\ 2.  May overlook social variables
    \\ 3.  May leads to unemployment
    \\ 4.  Inaccuracies are still possible
    \\ 5.  Susceptible to security risks
    \\\\AI has doubtless potential to improve healthcare systems, automating tedious tasks can free up clinician schedules to allow for more patient interfacing.
    \\ Real time data can better and more rapidly inform diagnoses. Whether a patient or physician, lives everywhere are improving thanks to AI.}
	\section{\underline{Making heavy machines portable}}
	\paragraph{Portable medical devices have revolutionized the way in which people monitor and determine their own health and wellbeing. These devices can’t just provide continuous and non-invasive monitoring of health parameters, but through connectivity they also provide real-time updates to healthcare providers.
	\\There is already a trend within the medical sector to make equipment more portable in the hospital; the ability to move diagnostic equipment to a patient's bedside instead of needing a dedicated room makes hospitals more efficient, while delivering higher levels of patient consultations. Also, the bulky machines can be connected to a decentralised server and a handy probe is used for reading and then the data will be uploaded on the server which will then give the detailed report accordingly. 
    \\Some of the widely used portable machines are:
    \\1.Blood Pressure Monitors: wireless BPMs are highly portable and they attach to your upper arm, use smart technology to record, and monitor your blood pressure.
    \\2.Glucose Monitoring Systems: smart, portable glucose monitoring systems let you measure the glucose levels in your blood on the go. The smart element is that these little devices can connect to your smartphone in order to monitor your blood sugar levels over time and share them with a doctor.
    \\The benefit of being able to test your glucose levels remotely is that you can do so before and after exercise, meals out, or in any situation that you might need to do so remotely without the usual blood test.
    \\3.Automated Insulin Pumps: people suffering from type 1 diabetes need to replenish their insulin stock with multiple daily injections or through a catheter.
    \\4.Portable EKG/ECG monitors assess heart activity used during electrocardiogram tests. While traditional models would display data as graphical lines on paper, the latest compact version includes LCD screens and sensors to let you view results in real time by taking your pulse via your fingers with an accompanying mobile application.
    \\5.Currently camera pills are also used for analysis, a pill that takes images of the inside of your digestive system – is being used to screen patients for bowel cancer from the comfort of their own homes.}
	\section{\underline{Blockchain in healthcare}}
	\paragraph{Blockchain has a wide range of applications and uses in healthcare. The ledger technology facilitates the secure transfer of patient medical records, manages the medicine supply chain and helps healthcare researchers unlock genetic code.
	\\\\First, let’s dive deeper into what blockchain really is.
    \\A blockchain is a distributed system that generates and stores data records. It maintains a digital ledger of connected “blocks” of information that represent how data is shared, changed or accessed on its peer-to-peer network.
    \\All devices on the same blockchain system will generate identical blocks when a connected device is involved in any kind of transaction. If one computer’s data is accessed, changed, shared or otherwise manipulated in any way, a block is generated to locally record that information on every device. This way, changes to data can be easily identified. It’s a decentralized approach that allows data parity to be achieved by comparing every connected device’s blocks.
    \\\\Benefits of blockchain is as follows:
    \\ 1.Security: Blockchain is verified through a consensus system and stored across many nodes, making DDoS cyberattacks and tampering with records extremely difficult.
    \\ 2.Cost efficiency: Middlemen who take a cut of transactions can be removed because consensus mechanisms create trust through transparency.
    \\ 3.Traceability: An immutable record of all transactions can reduce fraud and protect against liability.
    \\ 4.Business process speed: Automated smart contracts can reduce time of transactions because the process no longer requires manual oversight.
    \\ 5.Token value: Digital assets can hold virtual and real-world value, such as when a virtual token is used for a loyalty points programme.
    \\ 6.Confidentiality: Collaboration between organisations can occur without sharing sensitive information, e.g. individual medical records.
    \\ 7.Neutral and equal: No one company or individual owns the blockchain, encouraging trustworthiness and longevity of the system, e.g. if one of the founding parties leaves, the system will continue to work without them.
    \\\\However, with blockchain there are some demerits:
    \\ 1.Implementation and managinfg of blockchain requires deep knowledge, which needs experts to be hired.
    \\ 2.Blockchain can't go backwards, data is immutable, also data once inserted can't be deleted. Also using blockchain privacy is also at risk.
    \\ 3.Since blockchain is at its babystage thus cost of development isn't fixed,}
    \paragraph{\textit{\\\\At last, I also want to add that developments in healthcare is unlimited and as time passes, new technologies will came up that will revolutionise the whole heathcare system. Some of the technologies like quantum computing, nanoscale 3-D printing, artificial tissue, DNA computing \& storage, ontologies,  etc also takes healthcare to a great height like never thought before.}}
\end{document}